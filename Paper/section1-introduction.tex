%%%%%%%%%%%%%%%%%%%%%%%%%%%%%%%%%%%%%%%%%%%%%%%%%%%%%%%%%%%%%%%%%%%%
% Introduction
%%%%%%%%%%%%%%%%%%%%%%%%%%%%%%%%%%%%%%%%%%%%%%%%%%%%%%%%%%%%%%%%%%%%

\section{Introduction}
\label{sec:introduction}
Digital modelling and fabrication, which significantly bridge the gap between the 3D virtual design and physical product, attracts more and more attention in computer graphics community. A 3D assembly is composed of multiple component parts with specific form and/or functionality can be efficiently created in a modern CAD software. The software would give immediate feedback regarding user-defined property of the assembly. By editing the part geometry, designer could improve local performance of the assembly (such as part appearance, connection strength, etc.) with the help of feedback from the CAD software. However, local modification of part geometry is less effective for global properties (like structural stability and disassembling sequence of parts). Specific design tools are demanded by users.

This report focuses on 3D interlocking assemblies where all component parts have to be disassembled after removing a single key part. A general CAD software is inefficient for designing such interlocking assemblies because there is no obvious correspondence between the assembly geometry and disassembling sequence of parts, especially for non-professional designers. Though several computational methods for designing interlocking assemblies such as puzzle and furniture has recently been contributed, the interlocking mechanism has not yet been fully understood and the full search space of interlocking configurations has not been fully explored, restricting applicability for the design.

Designing a interlocking assembly, which sets partial constraints on disassembling sequence, is an inverse problem of finding a feasible disassembling sequence of parts for a given assembly. The latter is a classic research topics in geometric reasoning. Thus, I choose my first reference paper \cite{Wilson-1994-GeometricReasoning} about how to automatically generate a disassembly sequence and select a less complicate way of disassembling process. The input assembly is assumed to have three properties:

\begin{itemize}[leftmargin=*]
	\item Each part is rigid.
	\vspace*{1mm}
	\item Neighboring parts have planar surface contact only.
	\vspace*{1mm}
	\item Input assembly can be disassembled by single-part translational motions, i.e., part rotation is not required and all other parts remain fixed when removing a part.
\end{itemize}
The 3D model of an assembly from a typical CAD software are appropriate for rendering but it does not directly provide the information that is needed to easily plan assembly algorithms. Therefore, the authors propose a family of {\em Directional Blocking Graph} (DBGs), where each describes the potential interactions among parts for a given translation direction. It is proved only a finite subset of DBGs are pairwise different, which can be computed with a polynomial time complexity. To remove a part from an assembly, the in/out-degree of this part have to be zero in at least one DBG. Besides, for a given translation direction, any two parts have to be removed together as long as they are in the same strongly connected component(SCC) in the corresponding DBG. This powerful theory leads to the following fact.

\begin{displayquote}
	\em All directional blocking graphs(DBGs) of an interlocking assembly have to be strongly connected except the key.
\end{displayquote}
What is strongly connected? How to check strongly connected in a graph? How to make a graph strongly connected? The second reference paper \cite{Tarjan-1972-SCC} concludes these problems with efficient algorithms. By utilizing their graph analysis tools, test interlocking can be achieved in polynomial time complexity.

Interlocking has a long history in Chinese and Japanese furniture design but few types of artificial interlocking designs have been invented so far. Verifying interlocking property of an assembly requires testing the immobilization of all the subsets of pieces. Such tests lead to extremely expensive algorithms let alone designing an interlocking assemblies. To avoid exhaustively testing, my last reference paper \cite{Song-2012-InterCubes}, focus on a small subclass of interlocking puzzles that are recursive in the sense that the assembly of puzzle pieces (with at least three pieces) remains an interlocking puzzle also after the (sequential) removal of pieces. Still, they are the first who develop a computational method for generating new types of interlocking geometries. As a result, their method could design new voxelized puzzles with more pieces. 
\newpage
The section \ref{subsec: graphmodel} discuss the DBGs in detail and are followed by the section \ref{subsec: testinginterlocking} which presents the interlocking testing algorithm. The correctness of the recursive interlocking approach \cite{Song-2012-InterCubes},where maintaining  interlocking property in consecutive three parts make the assembly interlocking, can be proved by graph-based representation. Furthermore, there exists non-recursive interlocking which does not cover in \cite{Song-2012-InterCubes}. In addition, an assembly which part-graph have more than one cut-point,  cannot be interlocking. By understanding the augmentation problem from \cite{Tarjan-1972-SCC}, the section \ref{subsec: failurecases} gives one possibility to re-design the input geometry, such that its part-graph of is biconnected. 
