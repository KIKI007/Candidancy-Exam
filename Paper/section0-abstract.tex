\begin{abstract}
Generating a feasible disassembling sequence of a mechinical assembly is a fundamental research topic in geometric reasoning. Recently, researchers in computer graphics are interested in the inverse problem, which creates feasible geometry with respect to predefined constraints on the order of assembly. Among them, the interlocking assemblies that all component parts must be disassembled after a key part, have a long history in the design of puzzles, furniture, architecture, and other complex geometric structures. Though many design tools that allow creating interlocking assemblies has been recently contributed, the interlocking mechanism has not yet been understood which prevents exploring the full searching space as well as restricts their applicability for design. 

In this report, we propose a graph-based model for describing the interlocking mechanism. The core idea is to represent part relationships with a family of base {\em Directional Blocking Graphs (DBGs)}. With the help of classic graph theory, our approach build a connection between the connectivity of DBGs and its corresponding assembly's interlocking property. As a result, our model provide a easy guidebook for start-of-art interlocking design with more design flexibility.





%Interlocking assemblies have a long history in the design of puzzles, furniture, architecture, and other complex geometric structures. 
%The key defining property of interlocking assemblies is that all component parts are immobilized by their geometric arrangement, preventing the assembly from falling apart.
%Computer graphics research has recently contributed design tools that allow creating new interlocking assemblies.
%However, these tools focus on specific kinds of assemblies and explore only a limited  space of interlocking configurations, which restricts their applicability for design.
%
%In this paper, we propose a new general framework for designing interlocking assemblies.  The core idea is to represent part relationships with a family of base {\em Directional Blocking Graphs} and leverage efficient graph analysis tools to compute an interlocking arrangement of parts. This avoids the exponential complexity of brute-force search.  Our algorithm iteratively constructs the geometry of assembly components, taking advantage of all existing blocking relations when constructing successive parts. As a result, our approach supports a wider range of assembly forms compared to previous methods and provides significantly more design flexibility. 
%%
%%iteratively constructs the parts by finding required local part blocking relations to achieve interlocking and constructing parts/joints geometry that satisfy these requirements.
% %that describes blocking relations among parts for each specific moving direction.
% %
%We show that our framework facilitates efficient design of complex interlocking assemblies, including new solutions that cannot be achieved by state of the art approaches. 

\end{abstract}



