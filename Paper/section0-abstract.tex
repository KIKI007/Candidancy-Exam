\begin{abstract}
Computing a feasible disassembling sequence of parts for a given 3D assembly is a fundamental research topic in geometric reasoning. Its inverse problem which is to create geometry of a 3D assembly according to predefined constraints on the parts disassembly order, attracts more and more attention in computer graphics community. This report focuses on 3D interlocking assemblies where all component parts have to be disassembled after removing a single key part. Though several computational methods for designing interlocking assemblies such as puzzle and furniture has recently been contributed, the interlocking mechanism has not yet been fully understood and the full search space of interlocking configurations has not been fully explored, restricting applicability for the design.

In this report, I propose a graph-based method for modeling the interlocking mechanism. The core idea is to represent part blocking relationships with a family of base {\em Directional Blocking Graphs (DBGs)}. By utilizing graph analysis tools in classic graph theory, my approach builds a connection between an interlocking assembly and the connectivity of its DBGs. Based on this connection, I propose an efficient algorithm to test interlocking with polynomial time complexity and enable the ability to explore interlocking configurations that are not possible by the state of the art. As a result, my method has potential to lead to a more efficient and flexible computational tool for designing interlocking assemblies.
%Interlocking assemblies have a long history in the design of puzzles, furniture, architecture, and other complex geometric structures. 
%The key defining property of interlocking assemblies is that all component parts are immobilized by their geometric arrangement, preventing the assembly from falling apart.
%Computer graphics research has recently contributed design tools that allow creating new interlocking assemblies.
%However, these tools focus on specific kinds of assemblies and explore only a limited  space of interlocking configurations, which restricts their applicability for design.
%
%In this paper, we propose a new general framework for designing interlocking assemblies.  The core idea is to represent part relationships with a family of base {\em Directional Blocking Graphs} and leverage efficient graph analysis tools to compute an interlocking arrangement of parts. This avoids the exponential complexity of brute-force search.  Our algorithm iteratively constructs the geometry of assembly components, taking advantage of all existing blocking relations when constructing successive parts. As a result, our approach supports a wider range of assembly forms compared to previous methods and provides significantly more design flexibility. 
%%
%%iteratively constructs the parts by finding required local part blocking relations to achieve interlocking and constructing parts/joints geometry that satisfy these requirements.
% %that describes blocking relations among parts for each specific moving direction.
% %
%We show that our framework facilitates efficient design of complex interlocking assemblies, including new solutions that cannot be achieved by state of the art approaches. 

\end{abstract}



